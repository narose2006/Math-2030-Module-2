\documentclass{report}
\usepackage{graphicx} % Required for inserting images
\usepackage{amsmath}
\title{Math 2030 Module 2 Project: Gauss' Law}
\author{Noah` Rose}
\date{February 2026}

\begin{document}

\maketitle

\section{Introduction}
\ Electrostatics is not only a popular topic in physics, but also an important one. Electric fields are very in electronics. From the phones we use everyday, to powerful MRI's that allow doctors to see inside the human body, electrostatics is used constantly. However, nothing is every ideal, and the world is full of complex shapes, which causes issues when trying to figure out the relations for electric fields. That is where Gauss' Law comes in. Gauss' Law is used for finding expressions for the electric field, while not exactly using the electric field itself, but instead using the electric flux ("Applying Gauss's Law"). Gauss's Law does this through the use of a Gaussian Surface, an arbitrary shape in which certain symmetries can be used.This project intends to describe to the reader what Gauss' Law is, examples of it's uses in electric fields, and discuss how this is shown in the world around us. 
\section{What is Gauss' Law, and how is it formulated?}

\subsection{Electric Flux}
\ Before we talk about Gauss' Law, we first must talk about electric flux. Suppose we have an arbitrary object, which has a uniform electric field $\mathbf{E}$ passing through it. To make calculations simpler, we split the object into tiny square areas $\mathrm{dA}$, which have a normal vector pointing perpendicular to the area $\hat{\mathbf{n}}$. Thus, by definition, the electric flux is defined as $$\Phi_E=\int\mathbf{E}\cdot \mathrm{dA} \hat{\mathbf{n}} $$. Since we are summing an infinite amount of small areas, we integrate over $\mathrm{dA}$

\subsection{Gauss' Law}
\ Gauss' Law describes the relationship between an object and what's called as a Gaussian surface. A Gaussian surface is a surface which takes advantage of symmetries of it's charge distribution within an arbitrary object, such as spherical, cylindrical, or planar symmetries. From this, we can describe Gauss' Law as $$\Phi_E=\int \mathbf{E}\cdot\mathrm{dA}\hat{\mathbf{n}}=\frac{q_{encl}}{\epsilon_0}$$ 
\textit{Proof of Gauss' Law:}
Suppose we have Some arbitrary shaped object. We take the Gaussian Surface to be a Sphere with uniform charge $q_{encl}$ and radius r. Then the electric field going through the surface is $E=\frac{k_eq}{r^2]}$. To find the electric flux, we split the sphere into tiny areas $\mathrm{dA}$. Thus, the electric flux is $$\Phi_E=\int E\cdot\mathrm{dA}=EA=\frac{k_eq_{encl}}{r^2}\times4\pi r^2=\frac{q_{encl}}{4\pi\epsilon_0r^2}\times 4\pi r^2=\frac{q_{encl}}{\epsilon_0}$$, as required. Here by definition, coulomb's constant $k_e=\frac{1}{4\pi\epsilon_0}$
\subsection{How to use Gauss' Law}
The steps that are involved with Gauss's Law are quite simple to follow. First, find any symmetries in the problem (like cylindrical, etc...), then chose a Gaussian Surface that comes with the same symmetry as the problem states (For example, if a charge has spherical symmetries, use a sphere as the Gaussian Surface), then use Gauss' Law to evaluate the electric field with any charge distribution (Linear charge, surface charge, or volume charge density) ("Applying Gauss's). 

\section{Examples of how Gauss' Law can be used}
\ \textit{Example 1: Suppose we have an infinite rod of linear charge density $\lambda$ (That is, $\lambda=\frac{q}{l}$}), with radius r, what is the magnitude of the electric field at an arbitrary point p?

\ Solution: We take our gaussian surface to be a cylinder with radius r and length l. Since the electric field will be moving in the x direction, the only point in which the electric field acts on the gaussian surface is through the base of the cylinder. So: $$\Phi_E=\int E\cdot\mathrm{dA}=EA=\frac{q_{encl}}{\epsilon_0}$$

Since the area of a cylinder is $A=2\pi rL$, and $q_{encl}=\lambda L$, then: $$E=\frac{\lambda L}{A\epsilon_0}=\frac{\lambda L}{2\pi rL\epsilon_0}\longrightarrow E=\frac{\lambda}{2\pi r \epsilon_0}$$

\textit{Example 2: Consider a sphere with volume charge density $\rho=\frac{q}{V}$, and radius R. Find the magnitude of the electric field.}
\begin{figure}[!ht]
    \centering
    \includegraphics[width=0.85\linewidth]{Gauss's Law Example.png}
    \caption{A sphere with charge density $\rho$, and radius R With the gaussian surface inside and outside the sphere ("Applying Gauss's")}
    \label{fig:Fig 1}
\end{figure}

\ Solution: Choose the gaussian surface to be a sphere with radius r. Then we have two cases:
\\ Case 1: $r<R$. Then $q_{encl}=\rho(\frac{4}{3}\pi r^3)$. So: $$\Phi_E=\int E\cdot\mathrm{dA}=EA=\frac{q_{encl}}{\epsilon_0}$$.
$$E(4\pi r^2)=\frac{4\pi r^3\rho}{3\epsilon_0}\longrightarrow E=\frac{\rho r}{3\epsilon_0}$$
\\ Case 2: $r>R$. Then $q_{encl}=\rho(\frac{4}{3}\pi R^3)$, and: $$\Phi_E=EA=\frac{\rho(\frac{4}{3}\pi R^3)}{\epsilon_0}$$. $$E(4\pi r^2)=\frac{\rho(\frac{4}{3}\pi R^3)}{\epsilon_0}\longrightarrow E=\frac{\rho R^3}{3\epsilon_0 r^2}$$

\section{Real World Applications of Gauss's Law}
\subsection{Process Engineering Applications}
 Since Gauss's Law aids in finding electric field relations, it can be especially useful with anything involving electric fields. One such application is in process engineering. Electric fields, and subsequently Gauss's Law is used in electric field assisted combustion. Flames are electrically neutral, but when an electric field is applied, it can act as a cathode (Gains electrons), then positive ions from the electric field gets attracted towards the flame, which can affect chemical reactions, as well as the structure of the flame itself (Zigan 5). Another way is through what's called Electrostatic Precipitation (or ESP). The setup for this is electrodes, some of which are "Collecting" electrodes, and others are "discharge" electrodes which are uncharged. A high voltage is applied, which causes an electric field. This causes electric charge to attach to particles, which attracts the particles to the collecting electrode, which is then taken to be disposed of (Zigan 13). 

\subsection{Food Processing Applications}
\ The next application is with food processing. In this, a High Voltage Electric Field (HVEF) is used for many things, one of which being the removal of impurities in liquids. This is called electro-hydraulic discharge, and it makes use of High Voltage Electric Discharge (HVED), which is extremely good for heat transfer, and can remove impurities from common liquids like watter or oil, simply by heating it up enough so that the impurities are evaporated (Dalvi-isfahan et. al). 

\section{Conclusion}
\ Gauss's Law is both an important, and versatile law in the world of electrostatics, due to is usefulness in finding electric fields. From process engineering, to food processes, Gauss's Law can aid in finding the electric field of a given setup, which allows for accurate and precise measurements needed to successfully complete a task. By understanding how to use Gauss's Law, its uses are extremely efficient.   

\end{document}
