\documentclass{report}
\usepackage{graphicx} % Required for inserting images
\usepackage{amsmath}
\title{Math 2030 Module 2 Project: Gauss' Law}
\author{Noah` Rose}
\date{February 2026}

\begin{document}

\maketitle

\section{Introduction}
\ Electrostatics is not only a popular topic in physics, but also an important one. Electric fields are very in electronics. From the phones we use everyday, to powerful MRI's that allow doctors to see inside the human body, electrostatics is used constantly. However, nothing is every ideal, and the world is full of complex shapes, which causes issues when trying to figure out the relations for electric fields. That is where Gauss' Law comes in. Gauss' Law is used for finding expressions for the electric field, relating the electric flux to a closed surface (called a gaussian surface) and the charge enclosed within this surface (Kamberaj 27). This project intends to describe to the reader what Gauss' Law is, examples of it's uses in electric fields, and discuss how this is shown in the world around us. 
\section{What is Gauss' Law, and how is it formulated?}

\subsection{Electric Flux}
\ Before we talk about Gauss' Law, we first must talk about electric flux. Suppose we have an arbitrary object, which has a uniform electric field $\mathbf{E}$ passing through it. To make calculations simpler, we split the object into tiny square areas $\mathrm{dA}$, which have a normal vector pointing perpendicular to the area $\hat{\mathbf{n}}$. Thus, by definition, the electric flux is defined as $$\Phi_E=\int\mathbf{E}\cdot \mathrm{dA} \hat{\mathbf{n}} $$. Since we are summing an infinite amount of small areas, we integrate over $\mathrm{dA}$

\subsection{Gauss' Law}
\ Gauss' Law describes the relationship between an object and what's called as a Gaussian surface. A Gaussian surface is a surface which takes advantage of symmetries of it's charge distribution within an arbitrary object, such as spherical, cylindrical, or planar symmetries. From this, we can describe Gauss' Law as $$\Phi_E=\int \mathbf{E}\cdot\mathrm{dA}\hat{\mathbf{n}}=\frac{q_{encl}}{\epsilon_0}$$ 
\textit{Proof of Gauss' Law:}
Suppose we have Some arbitrary shaped object. We take the Gaussian Surface to be a Sphere with uniform charge $q_{encl}$ and radius r. Then the electric field going through the surface is $E=\frac{k_eq}{r^2]}$. To find the electric flux, we split the sphere into tiny areas $\mathrm{dA}$. Thus, the electric flux is $$\Phi_E=\int E\cdot\mathrm{dA}=EA=\frac{k_eq_{encl}}{r^2}\times4\pi r^2=\frac{q_{encl}}{4\pi\epsilon_0r^2}\times 4\pi r^2=\frac{q_{encl}}{\epsilon_0}$$, as required. Here by definition, columbs constant $k_e=\frac{1}{4\pi\epsilon_0}$

\end{document}
