\documentclass{report}
\usepackage{graphicx} % Required for inserting images

\title{Noah Rose: Math 2030 Research Proposal}
\author{Noah Rose}
\date{February 2026}

\begin{document}

\maketitle

\section{Introduction}
In the world of electricity, the concept of electric fields is a very important one, From powering electronics, to its use in medical imaging (Such as MRIs, etc...). However, simply defining an electric field is not always easy to do. In our world of imperfections, objects are just shaped too arbitrarily to use simple methods. That is until Gauss' Law, which allows us to define shapes that look complex into simple, ideal shapes, for which we can then use our desired methods to figure out the electric field. The purposes of Gauss' Law is not only to construct shapes, but to also relate the electric flux to the charge enclosed in this surface (Kamberaj 27). The planned research is to dive into Gauss' Law, how it works, and its applications in the world around us. 

\section{Key Research Questions}
Although there are many questions to be answered about this topic, the key ones are as follows:
\begin{enumerate}
    \item What is Gauss' Law?
    \item How is Gauss' Law helpful?
    \item How does this apply to real world scenarios?
\end{enumerate}
\section{Methodology}
The planned methodology for this research topic is quite simple, but effective. We start with Explaining the idea of Electric Flux, and the relation of Gauss' Law, Using a small, but effective proof. Next, examples of it's use in calculating the electric field of arbitrary shapes (Like an infinite rod of charge, or a conducting sphere) are given to gain understanding in how it is helpful. Finally, an Explanation of how Gauss' Law can be used in medical imaging, through the shapes of the machines not being simple shapes that do not require Gauss' Law).

\section{Github Link}
The link to the Github can be found here https://github.com/narose2006/Math-2030-Module-2/tree/b3970418583e31a13c2ce2f69a2d2166e4206d82

\section{References}
\begin{enumerate}
    \item Kamberaj, Hiqmet. \textit{Electromagnetism with solved problems}, Springer, 2022 
\end{enumerate}
\end{document}
